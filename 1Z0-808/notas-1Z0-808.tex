\documentclass[12pt]{report}
\usepackage[utf8]{inputenc} % Pacote para acentuação gráfica
%\usepackage[T1]{fontenc}
\usepackage[brazil]{babel} % nomes das estruturas em pt-br
%\usepackage{hyperref}
\usepackage{indentfirst} % indenta primeiro paragráfo após título
\usepackage{setspace} % pacote para alterar espaçamento entre linhas
%\setlength{\parindent}{1cm} % define o tamanho da indentação
%\setlength{\parskip}{0.3cm} % define o espaçamento vertical entre parágrafos

\usepackage{xcolor}
% Definindo novas cores
\definecolor{verde}{rgb}{0.25,0.5,0.35}
\definecolor{jpurple}{rgb}{0.5,0,0.35}
% Configurando layout para mostrar codigos Java
\usepackage{listings}
\lstset{
	language=Java,
	basicstyle=\ttfamily\small,
	keywordstyle=\color{jpurple}\bfseries,
	stringstyle=\color{red},
	commentstyle=\color{verde},
	morecomment=[s][\color{blue}]{/**}{*/},
	extendedchars=true,
	showspaces=false,
	showstringspaces=false,
	numbers=left,
	numberstyle=\tiny,
	breaklines=true,
	backgroundcolor=\color{cyan!10},
	breakautoindent=true,
	captionpos=b,
	xleftmargin=0pt,
	tabsize=4
}
\pagestyle{empty}
\begin{document}
\section*{Java Basics}
\subsection*{Define the scope of variables}
O escopo é o que determina em que pontos do código uma variável pode ser usada.
\subsubsection*{Variáveis locais}
Chamamos de locais as variáveis declaradas dentro de métodos ou construtores. Antes de continuar, vamos estabelecer uma regra básica: o ciclo de vida de uma variável local vai do ponto onde ela foi declarada até o fim do bloco onde ela foi declarada.

Mas o que é um bloco? Podemos entender como bloco um trecho de código entre chaves. Pode ser um método, um construtor, o corpo de um \textbf{\textit{if}}, de um \textbf{\textit{for}} etc.:

\begin{lstlisting}
public void m1() { // início do bloco do método
	int x = 10; // variável local do método
	
	if (x >= 10) { // início do bloco do if
		int y = 50; // variável local do if
		System.out.print(y);
		
	} // fim do bloco do if
	
} // fim do bloco do método
\end{lstlisting}

Analisando esse código, temos uma variável \textbf{\textit{x}}, que é declarada no começo do método. Ela pode ser utilizada durante todo o corpo do método. Dentro do \textbf{\textit{if}}, declaramos a variável \textbf{\textit{y}}. \textbf{\textit{y}} só pode ser utilizada dentro do corpo do \textbf{\textit{if}}, delimitado pelas chaves. Se tentarmos usar \textbf{\textit{y}} fora do corpo do \textbf{\textit{if}}, teremos um erro de compilação, pois a variável saiu do seu escopo.

Tome cuidado especial com loops \textbf{\textit{for}}. As variáveis declaradas na área de inicialização do loop só podem ser usadas no corpo do loop:
\pagebreak
\begin{lstlisting}
	
	for (int i = 0, j = 0; i < 10; i++)
	j++;
	
	System.out.println(j); // erro, já não está mais no escopo
	
\end{lstlisting}

\subsection*{Define the structure of a Java class}
\subsection*{Create executable Java applications with a main method; run a Java program from the command line; produce console output}
\subsection*{Import other Java packages to make them accessible in your code}
\subsection*{Compare and contrast the features and components of Java such as: platform independence, object orientation, encapsulation, etc.}

	
\end{document}